\documentclass[12pt]{article}
\input{/home/tritlo/Dropbox/Latex/header.tex}
\usepackage{listingsutf8}
\nonums

\title{Reiknigreind - Reikniverkefni 2}
\author{Matthías Páll Gissurarson}

\begin{document}
\maketitle

\section{Monte Carlo Spilari}

Niðurstöður:
Þegar \verb|Monte Carlo| spilaði við Random, þá vann \verb|Monte Carlo|  46 skipti af 60,
svo \verb|Monte Carlo|p hefur töluvert mikið forskot á móti random spilara.

Hinsvegar, þá fer \verb|Monte Carlo| vs. \verb|Monte Carlo Plus| $15-5-40$,
þ.e. \verb|Monte Carlo Plus| vann 40 skipti af 60, eða $2/3$, og $5$ jafntefli. Það er töluvert forskot,
en oftast þegar \verb|Monte Carlo| vann, þá var það vegna þess að hann var kominn með svikamyllu,
þ.e. þannig að sama hvað \verb|Monte Carlo Plus| gerði, þá mundi \verb|Monte Carlo| vinna.


\section{Heuristic}

Ég ákvað að prófa probabilisitc heuristic, sem metur hag gildi (utility value) ástands
útfrá því hve langt það er frá sigri/tapi, þ.e. gefur hátt utility fyrir skjótan sigur,
(og því ætti hann að velja það að blokka/sigra ef það er í boði),
en lægra fyrir sigur sem er lengra í burtu.
Einnig er gefin mikil refsing ef núverandi spilari er búinn að tapa,
en mikill ávinningur ef hann er búinn að vinna.

Þegar \verb|Heuristic| keppir við \verb|Monte Carlo|
þá sigrar það í $54$ skipti af $60$, sem er betra en gengið hjá \verb|Monte Carlo Plus|
Þegar það keppir við \verb|Monte Carlo Plus|, þá er staðan hinsvegar jöfn,
$30-30$, svo það virðist sem það sé jafn gott og \verb|Monte Carlo Plus|,
en það kemur heim og saman við það að \verb|Heuristic| er að hegða sér mjög líkt
\verb|Monte Carlo Plus|.

Þegar \verb|Heuristic| keppir við \verb|Board Inversion Heursitic|, þá fer það $57-3$ fyrir
\verb|Heuristic|, svo \verb|Heuristic| virðist vera betra mat á stöðunni en board inversion,
með $w$ sem $w$ sem gefið er í kynningunni.

Þegar \verb|Board Inversion Heuristic| keppir við \verb|Monte Carlo|, þá fer það $49-11$ fyrir
\verb|Monte Carlo|.


\section{Reinforcement Learning}

Við forritið var bætt við $TD(\lambda)$ útfærslu,
sem var unnin út frá $TD(0)$ matlab lausninni sem við fengum gefna.
Þegar $\lambda = 0$, þá hagar það sér eins og $TD(0)$, og skilar álíka win-rate og það gerði
fyrir $TD(0)$, sem bendir til þess að útfærslan sé rétt.
$TD(1)$ hagar sér eins og \verb|Monte Carlo| skv. fræðunum, en með $\lambda = 1$
fæ ég mjög álíkt winrate og \verb|Monte-Carlo| gegn random, en hinsvegar þegar að ég
læt það keppa við \verb|Monte Carlo|, þá virðist það ganga mun verr, en það sigrar
aðeins $13$ skipti af $60$. Það er sennilega vegna þess að í \verb|Monte-Carlo|
er ég að taka meðaltal útfrá stöðunni eins og hún er núna, en í \verb|Board Inversion Heuristic|
með $\lambda = 1$ er ég í raun ekki að taka mið út frá stöðunni eins og hún er, heldur
er ég  bara að byggja á því sem ég hef séð áður. Þegar $\lambda = 0.7$,
þá er þetta ennþá verra, en á móti \verb|Monte Carlo|, þá fer leikurinn
$11-3-46$, þ.e. \verb|Board Inversion Heuristic| sigrar $11$ leiki af $60$,
og það eru $3$ jafntefli. Það verður að teljas frekar slappt,
svo hugsanlega er galli í implementationinu á $TD(\lambda)$

\verb|Board Inversion Heuristic| mundi sennilega standa sig betur ef það væri þjálfað
í fleiri episode, en hjá mér eru það aðeins $2000$ epsiode sem það fer í gegnum,
en það er sökum þess hve lengi þetta tekur að keyra. Einnig væri hægt að tune-a
gildin á $\lambda, \alpha$ og $\gamma$ í forritinu til þess að fá betri niðurstöður.

\verb|N-tuples| voru implementuð á svipaðan hátt og þau voru gerð í sýnidæmi,
þ.e. þau voru generateu-ð þannig að þau væru eins og snákar í borðinu, en ekki
random eins og stungið var uppá. Þau voru síðan notuð til að þjálfa vigtir fyrir
þau tuple.

Þjálfun á spilaranum fyrir \verb|N-tuples| tók rosalega langan tíma,
en það er sennilega vegna þess að það var ekki útfært með fylkjum, heldur
með einfaldari hlutum í python. Það var því ekki þjálfað fyrir fleiri en $2000$ episode.

Þar sem að þetta eru heldur fá episode (talað er um 100 milljón episode 
í greininni), þá stóð \verb|N-tuples| sig frekar illa á móti \verb|Monte-Carlo Plus|
(selective útgáfunni), en staðan var $4-56$ \verb|Monte-Carlo Plus| í vil.
Þar sem að \verb|Monte-Carlo Plus| var mjög álíkt mínu Heuristic,
þá áætlum við að það muni fara svipað gegn því.

Það verður að teljast heldur slappt, en eins og bent hefur verið á, þá mundi
niðurstaðan vera mjög ólík ef þjálfuð væru fleiri episode. Sennilega
væri hægt að bæta þessa niðurstöðu með því að láta vigtirnar vera þjálfaðar
gegn \verb|Monte-Carlo Plus| spilara, frekar en random eins og gert er,
því þá mundi \verb|N-tuples| miða frekar að að vinna þann spilara.


\section{Hvað ég hef lært}
Eftir að hafa unnið þetta verkefni hef ég núna mun betri skilning á hvernig TD og Monte Carlo
hermanirnar virka, sérstaklega eftir að maður útfærði það sjálfur. Einnig veitti
útfærslan á N-tuples mér innsýn yfir í hvernig má búa til sniðugt subspace
af featurespace, til þess að geta ráðið við hversu stórt það verður.

\section{Viðauki}
\lstinputlisting[language=Python,breaklines=true]{../connect4.py}
\lstinputlisting[language=Python,breaklines=true]{../util.py}


\end{document}
